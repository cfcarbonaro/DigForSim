
\documentclass[10pt,a4paper]{article}
\usepackage[utf8]{inputenc}
%\usepackage[german]{babel}
\usepackage[T1]{fontenc}
\usepackage{amsmath}
\usepackage{amsfonts}
\usepackage{amssymb}
\usepackage{graphicx}
\usepackage{listings}
\usepackage{color}
\title{DigForSim}
\date{\today}
\usepackage[left=2cm,right=2cm,top=2cm,bottom=2cm]{geometry}
\begin{document}
\maketitle

\section{Create NavMesh}
Creating a working Navigation Mesh is hard work. You need experience, patience and luck.
Lets start:
\subsection{Import OSM-file to Blender}
\begin{enumerate}
\item Open new scene. delete default cube (and default plane).
\item Snap cursor to center.
\item Import buildings from OSM-file. Without extrusion!\\ \vspace{0.05cm}\\
\includegraphics[width=6cm]{../Pictures/osm_import.png} \\ 
\item[\textcolor{red}{Error}]  in \texttt{bb\_vorstadt.osm}: about 4 buildings were extruded (despite the fact, that I unchecked the extrusion by levels box too). I deleted those buildings.
\item Extrude now, from zero to five in z-direction (5m).
\item \textit{Edit-mode}. Select all. Mesh-> clean up-> remove doubles
\item \textit{Edit-mode}. Select all. Mesh-> clean up-> delete loose
\item \textit{Add Plane}, \textit{scale} it. 
\item Grab (g) + z 0.5m negative direction (-0.5).
\item Extrude upper surface to +0.1795 (or simply +0.2). (Slightly overlapping the :buildings objects.)
\item[\textcolor{green}{Important}] Apply scale to plane! If not, vertices will have other measure than :building-vertices. 
\item Select :buildings first and -> operators -> boolean -> Union with plane
\item[(optional)] maybe also possible the other way around: first select plane, then buildings 
\item Move rest to other layer...
\end{enumerate}
\subsection{Clear Topology}
Now the next part, which is very individual. 
\begin{enumerate}
\item \textit{Edit-mode}. Select all. Mesh-> clean up-> remove doubles (again)
\item \textit{Edit-mode}. Select all. Mesh-> clean up-> delete loose (again)
\item Search for anomalities in topology of the new mesh. (e.g. I found single edges in \texttt{bb\_vorstadt.osm}.
Or here in \texttt{vicoli.osm} is left of the cursor a redundant edge:\\ \vspace{0.05cm}\\ 
\includegraphics[width=10cm]{../Pictures/edge_error.png} \\ \vspace{0.05cm}\\
Therefore, check the whole surface of the new mesh. Delete the redundant edges.
\item When sure, that everything is clean, select new mesh -> scene tab -> create Navmesh
\item There are many parameters, don't know them well. -> \textit{radius} to 0.20 (makes agents go through narrow streets too). \\ \vspace{0.05cm}\\ 
\includegraphics[width=10cm]{../Pictures/screenshot_navmesh.png} 
\item Recalculate Indices (button on pyhsics tab)\\ \vspace{0.05cm}\\ 
\includegraphics[width=10cm]{../Pictures/screenshot_navmesh_2.png}
\item The Navmesh should now look something like this (with edge-selection in Edit-mode):\\ \vspace{0.05cm}\\ 
\includegraphics[width=10cm]{../Pictures/screenshot_navmesh_3.png} 

\end{enumerate}

\subsection{Test: Setting agent manually on NavMesh}
\begin{enumerate}
\item Add e.g. two \textit{spheres} (could be anything) to the scene
\item Example game-logic for first agent (Sphere):\\ \vspace{0.05cm}\\ 
\includegraphics[width=\textwidth]{../Pictures/game_logic.png} \\ \vspace{0.05cm}\\ 
where the target is the second agent (Sphere.001) and the NavMesh's name is Navmesh.
Sensor and Actuator are connected via And-Controller. Later, here will be a Python-Controller for more advanced logic. After starting the game enmgine it looks like this (red line because of Visualize-checkbox): \\ \vspace{0.05cm}\\ 
\includegraphics[width=10cm]{../Pictures/game_test.png}\\ \vspace{0.05cm}\\
\item Seems to work! If it looks like this, one can start to develop the simulation.

\end{enumerate}


\end{document}
